\documentclass{article}\usepackage[]{graphicx}\usepackage[]{color}
%% maxwidth is the original width if it is less than linewidth
%% otherwise use linewidth (to make sure the graphics do not exceed the margin)
\makeatletter
\def\maxwidth{ %
  \ifdim\Gin@nat@width>\linewidth
    \linewidth
  \else
    \Gin@nat@width
  \fi
}
\makeatother

\definecolor{fgcolor}{rgb}{0.345, 0.345, 0.345}
\newcommand{\hlnum}[1]{\textcolor[rgb]{0.686,0.059,0.569}{#1}}%
\newcommand{\hlstr}[1]{\textcolor[rgb]{0.192,0.494,0.8}{#1}}%
\newcommand{\hlcom}[1]{\textcolor[rgb]{0.678,0.584,0.686}{\textit{#1}}}%
\newcommand{\hlopt}[1]{\textcolor[rgb]{0,0,0}{#1}}%
\newcommand{\hlstd}[1]{\textcolor[rgb]{0.345,0.345,0.345}{#1}}%
\newcommand{\hlkwa}[1]{\textcolor[rgb]{0.161,0.373,0.58}{\textbf{#1}}}%
\newcommand{\hlkwb}[1]{\textcolor[rgb]{0.69,0.353,0.396}{#1}}%
\newcommand{\hlkwc}[1]{\textcolor[rgb]{0.333,0.667,0.333}{#1}}%
\newcommand{\hlkwd}[1]{\textcolor[rgb]{0.737,0.353,0.396}{\textbf{#1}}}%

\usepackage{framed}
\makeatletter
\newenvironment{kframe}{%
 \def\at@end@of@kframe{}%
 \ifinner\ifhmode%
  \def\at@end@of@kframe{\end{minipage}}%
  \begin{minipage}{\columnwidth}%
 \fi\fi%
 \def\FrameCommand##1{\hskip\@totalleftmargin \hskip-\fboxsep
 \colorbox{shadecolor}{##1}\hskip-\fboxsep
     % There is no \\@totalrightmargin, so:
     \hskip-\linewidth \hskip-\@totalleftmargin \hskip\columnwidth}%
 \MakeFramed {\advance\hsize-\width
   \@totalleftmargin\z@ \linewidth\hsize
   \@setminipage}}%
 {\par\unskip\endMakeFramed%
 \at@end@of@kframe}
\makeatother

\definecolor{shadecolor}{rgb}{.97, .97, .97}
\definecolor{messagecolor}{rgb}{0, 0, 0}
\definecolor{warningcolor}{rgb}{1, 0, 1}
\definecolor{errorcolor}{rgb}{1, 0, 0}
\newenvironment{knitrout}{}{} % an empty environment to be redefined in TeX

\usepackage{alltt}
\usepackage[margin=0.5in]{geometry}
\usepackage[utf8]{inputenc}
\IfFileExists{upquote.sty}{\usepackage{upquote}}{}
\begin{document}

The European Parliament (EP), despite its status as the most recognized EU institution \cite[e.g.,]{eurobarometer.2014}, still struggles to communicate its mission and relevance to EU citizens \citep*{anderson.mcleod.2004}. Turnout for EP elections remains low \citep*{hobolt.2014, franklin.hobolt.2011, mattila.2013}, citizens tend to see EP elections as "second-order" events \citep*{hix.marsh.2007, schmitt.2005}, and Europeans routinely fail to follow and recall EP-related news \citep*{eurobarometer.2013}. 

For all of these reasons, one might expect members of the European Parliament (MEPs) to embrace Twitter and other Web 2.0 technologies. Theoretically, these technologies could help MEPs to narrow the gap between themselves and citizens, to increase citizens' sense of political efficacy, and to publicize MEPs' work to audiences that matter (e.g., national party elites, organizational supporters, citizens at large). While Twitter has become an increasingly mainstream tool of political campaigning and has been adopted and used by incumbent politicians across a wide range of national contexts, this paper departs from the observation that there is significant variation in the extent and nature of MEPs' Twitter use. We demonstrate and seek to explain the significant variation that characterizes MEPs Twitter use.

We pursue three specific lines of inquiry. First, we  

The literature on politicians' use of Twitter contains at least two discernible streams. One stream examines the ways that parties and candidates use Twitter during election campaigns. In the EP context, for example, Vergeer, Hermans, and Sams (\citeyear{vergeer.hermans.sams.2011,vergeer.hermans.sams.2013}), have scrutinized the ways that Dutch parties and candidates used Twitter in the 2009 EP campaign. A second stream involves scrutiny of the ways that national legislators, once in office, use Twitter to interact with various interlocutors \citep*{ausserhofer.maireder.2013, chi.yang.2010, glass man.strauss.shogan.2013, larsson.kalsnes.2014, peterson.2012, shogun.2010, williams.gulati.2010}. To the best of our knowledge, however, researchers have yet to present a systematic analysis of MEPs' use of Twitter. This paper attempts to fill that gap.

*We are interested in two issues--adoption/use and "quality" of Twitter interactions. Politicians can use Twitter two ways - BROADCAST vs. CHAT.

The majority, 69\%, of MEPs have a presence on Twitter. 
% latex table generated in R 3.1.2 by xtable 1.7-4 package
% Tue Feb  3 13:12:24 2015
\begin{table}[ht]
\centering
\begin{tabular}{rr}
  \hline
on\_twitter & off\_twitter \\ 
  \hline
516 & 235 \\ 
   \hline
\end{tabular}
\end{table}

Number of MEPs on twitter, by country. The table is sorted by the percentage of MEPs for a country that aren't on Twitter. It would be interesting to investigate further what is causing the adoption rates here to vary. We could start to incorporate country level statistics into our data.

The countries twitter adoption rate has no impact on the overall number of tweets, evident in the plot below.

% latex table generated in R 3.1.2 by xtable 1.7-4 package
% Tue Feb  3 13:12:24 2015
\begin{table}[ht]
\centering
\begin{tabular}{lrrrr}
  \hline
nationality & on\_twitter & off\_twitter & percent\_on & tweets \\ 
  \hline
Malta & 6 & 0 & 100 & 1164 \\ 
  Netherlands & 24 & 2 & 92 & 9989 \\ 
  Ireland & 10 & 1 & 91 & 5078 \\ 
  Italy & 65 & 8 & 89 & 18329 \\ 
  Latvia & 7 & 1 & 88 & 1467 \\ 
  Slovenia & 7 & 1 & 88 & 1718 \\ 
  Finland & 11 & 2 & 85 & 2388 \\ 
  United Kingdom & 61 & 12 & 84 & 28181 \\ 
  Belgium & 17 & 4 & 81 & 2556 \\ 
  Sweden & 16 & 4 & 80 & 6208 \\ 
  Denmark & 10 & 3 & 77 & 1481 \\ 
  France & 56 & 18 & 76 & 13057 \\ 
  Spain & 40 & 14 & 74 & 18204 \\ 
  Austria & 13 & 5 & 72 & 6043 \\ 
  Czech Republic & 15 & 6 & 71 & 3113 \\ 
  Cyprus & 4 & 2 & 67 & 276 \\ 
  Croatia & 7 & 4 & 64 & 511 \\ 
  Poland & 29 & 22 & 57 & 7412 \\ 
  Germany & 54 & 42 & 56 & 9923 \\ 
  Slovakia & 7 & 6 & 54 & 442 \\ 
  Greece & 11 & 10 & 52 & 2376 \\ 
  Estonia & 3 & 3 & 50 & 384 \\ 
  Luxembourg & 3 & 3 & 50 & 264 \\ 
  Romania & 16 & 16 & 50 & 1141 \\ 
  Bulgaria & 8 & 9 & 47 & 1549 \\ 
  Lithuania & 4 & 7 & 36 & 121 \\ 
  Hungary & 7 & 14 & 33 & 640 \\ 
  Portugal & 5 & 16 & 24 & 1816 \\ 
   \hline
\end{tabular}
\end{table}

Here is the same table as above except it examines rates amount European Party. It is pretty clear here that the left is more likely to be on Twitter than the right. 

% latex table generated in R 3.1.2 by xtable 1.7-4 package
% Tue Feb  3 13:12:24 2015
\begin{table}[ht]
\centering
\begin{tabular}{lrrrr}
  \hline
european\_party & on\_twitter & off\_twitter & percent\_on & tweets \\ 
  \hline
Group of the Greens/European Free Alliance & 44 & 6 & 88 & 16935 \\ 
  Group of the Alliance of Liberals and Democrats fo... & 51 & 17 & 75 & 16939 \\ 
  Confederal Group of the European United Left - Nor... & 38 & 14 & 73 & 13377 \\ 
  Group of the Progressive Alliance of Socialists an... & 135 & 56 & 71 & 35400 \\ 
  Group of the European People's Party (Christian De... & 147 & 73 & 67 & 30680 \\ 
  Europe of Freedom and Direct Democracy Group & 31 & 17 & 65 & 15344 \\ 
  European Conservatives and Reformists Group & 45 & 25 & 64 & 9090 \\ 
  Non-attached Members & 25 & 27 & 48 & 8066 \\ 
   \hline
\end{tabular}
\end{table}

These are the most profilic users during the timeframe of our study. It's interesting because this doesn't perfectly line up with the overall most prolific users from what I've seen. I'm having an issue with the date format from age. I will add it to this table as soon as I resolve the issue.

% latex table generated in R 3.1.2 by xtable 1.7-4 package
% Tue Feb  3 13:12:24 2015
\begin{tabular}{llrrrrr}
  \hline
name & nationality & age & tweets & retweets & status\_replies & name\_replies \\ 
  \hline
David COBURN & United Kingdom & 2960 & 2828.00 & 140319.00 & 662.00 & 669.00 \\ 
  Lara COMI & Italy & 885 & 1886.00 & 13709.00 & 297.00 & 307.00 \\ 
  Margot PARKER & United Kingdom & 2243 & 1745.00 & 76835.00 & 234.00 & 235.00 \\ 
  Luke Ming FLANAGAN & Ireland & 285 & 1654.00 & 61208.00 & 540.00 & 596.00 \\ 
  Michel REIMON & Austria & 746 & 1604.00 & 37927.00 & 531.00 & 536.00 \\ 
  Fredrick FEDERLEY & Sweden & -5145 & 1550.00 & 44445.00 & 818.00 & 826.00 \\ 
  Raffaele FITTO & Italy & 2360 & 1515.00 & 6019.00 & 113.00 & 129.00 \\ 
  Javier COUSO PERMUY & Spain & 2118 & 1449.00 & 81965.00 & 225.00 & 225.00 \\ 
  Jacek SARYUSZ-WOLSKI & Poland & 2199 & 1432.00 & 50038.00 & 111.00 & 112.00 \\ 
  Nathan GILL & United Kingdom & -970 & 1416.00 & 194593.00 & 109.00 & 124.00 \\ 
   \hline
\end{tabular}




There isn't a strong correlation between retweets and overall tweets. The red line shows a regression fit line and the blue shows LOWESS. It's important to note that retweets are a point in time variable. This variable could change if we went back to the API and got the tweet again. For example, if we captured a tweet in the morning and it was retweeted heavily in the afternoon, we would not capture those retweets.

I am interested in this relationship and it might be a point that we investigate further. We could potentially pick a small subset of tweets by MEPs, and track them individually over time to see how they evolve. Since this would be a small subset, we could also likely track details about the individual users that retweeted the tweet and profile them.

\begin{knitrout}
\definecolor{shadecolor}{rgb}{0.969, 0.969, 0.969}\color{fgcolor}

{\centering \includegraphics[width=\maxwidth]{figure/twitter_adoption_user_plot-1} 

}



\end{knitrout}

This table shows the most prolific repliers. 

% latex table generated in R 3.1.2 by xtable 1.7-4 package
% Tue Feb  3 13:12:25 2015
\begin{tabular}{llrrrrr}
  \hline
name & nationality & age & tweets & retweets & status\_replies & name\_replies \\ 
  \hline
Fredrick FEDERLEY & Sweden & -5145 & 1550.00 & 44445.00 & 818.00 & 826.00 \\ 
  David COBURN & United Kingdom & 2960 & 2828.00 & 140319.00 & 662.00 & 669.00 \\ 
  Jörg LEICHTFRIED & Austria & 2239 & 1060.00 & 8932.00 & 650.00 & 662.00 \\ 
  Luke Ming FLANAGAN & Ireland & 285 & 1654.00 & 61208.00 & 540.00 & 596.00 \\ 
  Michel REIMON & Austria & 746 & 1604.00 & 37927.00 & 531.00 & 536.00 \\ 
  Julia REDA & Germany & 558 & 1145.00 & 44813.00 & 416.00 & 422.00 \\ 
  Nessa CHILDERS & Ireland & 2129 & 1142.00 & 11305.00 & 382.00 & 407.00 \\ 
  Patrick O'FLYNN & United Kingdom & 2394 & 1206.00 & 55913.00 & 378.00 & 386.00 \\ 
  Tomáš ZDECHOVSKÝ & Czech Republic & -6434 & 819.00 & 7043.00 & 358.00 & 378.00 \\ 
  Bill ETHERIDGE & United Kingdom & 689 & 652.00 & 5987.00 & 352.00 & 359.00 \\ 
   \hline
\end{tabular}


Here are the total number of tweets plotted against the number of replies. There are two type of replies, replies to statuses and replies to screen names. A reply to a screen name is always a reply to a status. 

\begin{knitrout}
\definecolor{shadecolor}{rgb}{0.969, 0.969, 0.969}\color{fgcolor}

{\centering \includegraphics[width=\maxwidth]{figure/twitter_cor_tweets_replies-1} 

}



\end{knitrout}
\end{document}
